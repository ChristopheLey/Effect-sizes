\documentclass[man]{apa6}
\usepackage{lmodern}
\usepackage{amssymb,amsmath}
\usepackage{ifxetex,ifluatex}
\usepackage{fixltx2e} % provides \textsubscript
\ifnum 0\ifxetex 1\fi\ifluatex 1\fi=0 % if pdftex
  \usepackage[T1]{fontenc}
  \usepackage[utf8]{inputenc}
\else % if luatex or xelatex
  \ifxetex
    \usepackage{mathspec}
  \else
    \usepackage{fontspec}
  \fi
  \defaultfontfeatures{Ligatures=TeX,Scale=MatchLowercase}
\fi
% use upquote if available, for straight quotes in verbatim environments
\IfFileExists{upquote.sty}{\usepackage{upquote}}{}
% use microtype if available
\IfFileExists{microtype.sty}{%
\usepackage{microtype}
\UseMicrotypeSet[protrusion]{basicmath} % disable protrusion for tt fonts
}{}
\usepackage{hyperref}
\hypersetup{unicode=true,
            pdftitle={Skewness and kurtosis: relation between Cain et al.~(2017) and the package `PearsonDS'},
            pdfauthor={Marie Delacre},
            pdfkeywords={keywords},
            pdfborder={0 0 0},
            breaklinks=true}
\urlstyle{same}  % don't use monospace font for urls
\usepackage{graphicx,grffile}
\makeatletter
\def\maxwidth{\ifdim\Gin@nat@width>\linewidth\linewidth\else\Gin@nat@width\fi}
\def\maxheight{\ifdim\Gin@nat@height>\textheight\textheight\else\Gin@nat@height\fi}
\makeatother
% Scale images if necessary, so that they will not overflow the page
% margins by default, and it is still possible to overwrite the defaults
% using explicit options in \includegraphics[width, height, ...]{}
\setkeys{Gin}{width=\maxwidth,height=\maxheight,keepaspectratio}
\IfFileExists{parskip.sty}{%
\usepackage{parskip}
}{% else
\setlength{\parindent}{0pt}
\setlength{\parskip}{6pt plus 2pt minus 1pt}
}
\setlength{\emergencystretch}{3em}  % prevent overfull lines
\providecommand{\tightlist}{%
  \setlength{\itemsep}{0pt}\setlength{\parskip}{0pt}}
\setcounter{secnumdepth}{0}
% Redefines (sub)paragraphs to behave more like sections
\ifx\paragraph\undefined\else
\let\oldparagraph\paragraph
\renewcommand{\paragraph}[1]{\oldparagraph{#1}\mbox{}}
\fi
\ifx\subparagraph\undefined\else
\let\oldsubparagraph\subparagraph
\renewcommand{\subparagraph}[1]{\oldsubparagraph{#1}\mbox{}}
\fi

%%% Use protect on footnotes to avoid problems with footnotes in titles
\let\rmarkdownfootnote\footnote%
\def\footnote{\protect\rmarkdownfootnote}


  \title{Skewness and kurtosis: relation between Cain et al.~(2017) and the package `PearsonDS'}
    \author{Marie Delacre\textsuperscript{1}}
    \date{}
  
\shorttitle{G1 and G2}
\affiliation{
\vspace{0.5cm}
\textsuperscript{1} Service of Analysis of the Data, Université Libre de Bruxelles, Belgium}
\keywords{keywords\newline\indent Word count: X}
\usepackage{csquotes}
\usepackage{upgreek}
\captionsetup{font=singlespacing,justification=justified}

\usepackage{longtable}
\usepackage{lscape}
\usepackage{multirow}
\usepackage{tabularx}
\usepackage[flushleft]{threeparttable}
\usepackage{threeparttablex}

\newenvironment{lltable}{\begin{landscape}\begin{center}\begin{ThreePartTable}}{\end{ThreePartTable}\end{center}\end{landscape}}

\makeatletter
\newcommand\LastLTentrywidth{1em}
\newlength\longtablewidth
\setlength{\longtablewidth}{1in}
\newcommand{\getlongtablewidth}{\begingroup \ifcsname LT@\roman{LT@tables}\endcsname \global\longtablewidth=0pt \renewcommand{\LT@entry}[2]{\global\advance\longtablewidth by ##2\relax\gdef\LastLTentrywidth{##2}}\@nameuse{LT@\roman{LT@tables}} \fi \endgroup}


\DeclareDelayedFloatFlavor{ThreePartTable}{table}
\DeclareDelayedFloatFlavor{lltable}{table}
\DeclareDelayedFloatFlavor*{longtable}{table}
\makeatletter
\renewcommand{\efloat@iwrite}[1]{\immediate\expandafter\protected@write\csname efloat@post#1\endcsname{}}
\makeatother
\usepackage{lineno}

\linenumbers

\authornote{

Correspondence concerning this article should be addressed to Marie Delacre, CP191, avenue F.D. Roosevelt 50, 1050 Bruxelles. E-mail: \href{mailto:marie.delacre@ulb.ac.be}{\nolinkurl{marie.delacre@ulb.ac.be}}}

\abstract{

}

\begin{document}
\maketitle

In 2017, Cain et al.~have conducted a review assessing the skewness and kurtosis of articles in recent psychology and education publications. They used the following formulas of Fisher's skewness (\(G_{1}\)) and kurtosis (\(G_{2}\)):

\begin{equation} 
G_{1}=\frac{\sqrt{n(n-1)}}{n-2} \frac{m_{3}}{\sqrt{(m_{2})^3}}
\label{eq:skew}
\end{equation}

With s = standard deviation, n = sample size, \(m_{2}\) = second centered moment and \(m_{3}\) = third centered moment.

\begin{equation} 
G_{2}=\frac{n-1}{(n-2)(n-3)}\times [(n+1)(\frac{m_{4}}{(m_{2})^2}-3)+6]
\label{eq:kurt}
\end{equation}

With s = standard deviation, n = sample size and \(m_{3}\)=third centered moment.

I chose to use this article in order to define which value of skewness and kurtosis I would simulate, in order to test the goodness of different measures of effect sizes under realistic population parameter values. In my simulations, I Chose the function \enquote{rPearson} from the package \enquote{PearsonDS}, in which skewness and kurtosis are computed as following:

\begin{equation} 
skewness=\frac{m_{3}}{\sqrt{(m_{2})^3}}
\label{eq:skewnessrPearson}
\end{equation}

\begin{equation} 
kurtosis=\frac{m_{4}}{(m_{2})^2}
\label{eq:kurtosisrPearson}
\end{equation}

In order to simulate a sample extracted from a population where \(G_{1}= X\), using the \enquote{rPearson} function, I need to make the following transformation:

\begin{equation} 
\frac{\sqrt{n(n-1)}}{n-2} \frac{m_{3}}{\sqrt{(m_{2})^3}} = X
<==> \frac{m_{3}}{\sqrt{(m_{2})^3}} = \frac{X(n-2)}{\sqrt{n(n-1)}}
\label{eq:skewnesstransformation}
\end{equation}

In order to simulate a sample extracted from a population where \(G_{2}= X\), using the \enquote{rPearson} function, I need to make the following transformation:

\begin{equation} 
\frac{n-1}{(n-2)(n-3)} [(n+1)(\frac{m_{4}}{(m_{2})^2}-3)+6] = X  
<==> \frac{m_{4}}{(m_{2})^2}=\frac{X(n-2)(n-3)-6(n-1)}{n^2-1}+3
\label{eq:kurtosistransformation}
\end{equation}


\end{document}
